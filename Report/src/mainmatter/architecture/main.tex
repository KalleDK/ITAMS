\chapter{Architecture}

We have defined three blocks that outlines our system, as seen in the Block Definition Diagram in Figure~\ref{fig:bdd}. The official responsibilities of the blocks in Figure~\ref{fig:bdd} is as follows:

\begin{itemize}
	\item \textbf{Mega32}: Runs the game, processes input from the controller and draws game on the display.
	\item \textbf{Display}: Displays the game.
	\item \textbf{Controller}: Controls the game.
	\item \textbf{3.3V Circuit}: Ensures voltage level and logic level is 3.3V.
\end{itemize}

\begin{figure}
\centering
\includegraphics[width=0.60\textwidth, trim={5mm 5mm 5mm 5mm}, clip, page=1]{architecture/bdd_ibd}
\caption{Block Definition Diagram}
\label{fig:bdd}
\end{figure}

We've identified the Display as a PCD8544. We've also found out that the NES controller uses a CD/TC4021 (8-stage static shift register) to store the state of the controller. Therefore, we'll have write at least two drivers for our system: one to interface the display and one to interface the controller.

We have chosen to describe our system using an Internal Block Diagram (IBD) as seen in Figure~\ref{fig:ibd}. The IBD shows the connections between the blocks.

\begin{figure}
\centering
\includegraphics[width=0.65\textwidth, trim={10mm 5mm 5mm 5mm}, clip, page=2]{architecture/bdd_ibd}
\caption{Internal Block Diagram}
\label{fig:ibd}
\end{figure}

\chapter{Implementation}\label{cha:implementation}

\chapter{Implementation}\label{cha:implementation}

\chapter{Implementation}\label{cha:implementation}

\input{mainmatter/implementation/drivers/main}
\chapter{Implementation}\label{cha:implementation}

\chapter{Implementation}\label{cha:implementation}

\chapter{Implementation}\label{cha:implementation}

\input{mainmatter/implementation/drivers/main}
