\section{Display}

The driver for the PD8544 is our way to interface the display. We will implement this driver by using key data from the data sheet on the chip, we've found.

The PD8544 has eight connections available to us\cite{philips:pcd8544}:
\begin{itemize}
\item \texttt{Data/Control}
\item \texttt{Clock Enable}
\item \texttt{Clock}
\item \texttt{Data In}
\item \texttt{Reset}
\item \texttt{Backlight}
\item \texttt{VCC}
\item \texttt{GND}
\end{itemize}

We will need to connect to most of these and interface them correctly in order to use the display. The PD8544 uses a serial interface, which means we are only to send one bit at a time.

The \texttt{Data/Control} connection to select whether we are sending control commands or data to the display. The connection has to be \texttt{HIGH} for data and \texttt{LOW} for commands\cite{philips:pcd8544}.

The \texttt{Clock Enable} connection is used to tell the display, that we have initiated communication and \texttt{Data in} is used to send data. The data is clocked using the \texttt{Clock} connection, where we provide a serial clock.

The \texttt{Reset} connection is used to reset the display. The rest of the connections are used for power to the display and the back light.

Since the chip uses a serial interface, we have decided to use the ATMEGA32's SPI (Serial Peripheral interface). 
Since we are only able to send data to the display, not receive, we only have use for the \texttt{MOSI} (Master Out, Slave In) connection, and not the \texttt{MISO} (Master In, Slave Out) connection.

The display runs on 3.3V. Since our microcontroller runs on 5V, we have to deal with this. 
We could either make a little circuit to convert the logic level from 5V to 3.3V. Or we could just ignore it, since the display is able to handle a supply of up to 7V. It will, however, run best at 3.3V.

Our plan is to create two drivers: a SPI driver and a Display driver.

\subsection{SPI driver}

The SPI interface of the ATMEGA32 is pretty straight forward. There is a few parameters, we need to keep in mind:

\begin{itemize}
\item SPI mode
\item Clock frequency
\item Endianness
\end{itemize}

These properties will have to match the properties defined in the data sheet of the Display. The implementation will be described in Chapter~\ref{cha:implementation}.

\subsection{Display driver}

The display driver is using the SPI driver for communicating with the display. 

This driver handles the trivial setup of the screen, so it can be used without further setup by the other modules in our system.

This driver implements the different display functions, and makes them available for use, such as setting the X and Y address and changing addressing mode. 

The driver, also, controls the \texttt{Reset} and \texttt{Data/Control} connections to ensure the correct usage.

The implementation will be described in Chapter~\ref{cha:implementation}.