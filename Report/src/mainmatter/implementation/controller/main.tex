\section{Controller}

This section describes the controller implementation. The implementation consists of a SPI driver and a Controller driver.

\subsection{SPI driver}

Since the hardware SPI driver is in use, we decided to write a SPI driver written in software. This is not a fully fledged SPI Driver as it is only Master In, Slave Out.

\subsubsection{Setup}

On setup, the software SPI driver needs to know: 
\begin{itemize}
	\item which type of buffer it should use to return the data in
	\item how many bits it should read
	\item which pins should be used as I/O
	\item which serial clock and sample mode it should use and the four timing values
\end{itemize}

The \texttt{SPI::Buffer} type can be used as a buffer. The buffer can be setup with the amount of bits it should hold, and using that number it calculates  the most optimal container to keep the bits in.

There are three I/O pins: \texttt{Data In}, \texttt{Latch} and \texttt{Clock}. In the I/O setting, it can be chosen which pins they should be using.

The mode type should be given a clock polarity value and a clock phase value. This helps eliminating and choose functions for the transfers (see Listing~\ref{lst:cpol_pulse}).

\begin{lstlisting}[caption={CPOL Determines when to sample},label={lst:cpol_pulse},frame=tlrb, language=C++]{Name}
template<typename Q = MODE>
inline typename enable_if<(Q::cpha == 0), void>::type
pulse(buffer_type& buffer) {
    buffer.add_bit(read_bit());
  	mark();
  	space();
}

template<typename Q = MODE>
inline typename enable_if<(Q::cpha == 1), void>::type
pulse(buffer_type& buffer) {
    mark();
  	buffer.add_bit(read_bit());
  	space();
}
\end{lstlisting}

The four timing values are:
\begin{itemize}
	\item Min. Removal Time - From LATCH is lowered until CLOCK may be risen
	\item Min. Pulse Width - From LATCH is raised until LATCH may be lowered
	\item Min. Hold Time - From CLOCK is raised until DATA can be read
	\item Min. Set-up Time - Setup time for next DATA
\end{itemize}

\subsection{Controller driver}

The NES Controller uses the software SPI driver to read the state of the buttons on the controller. The function \texttt{update} is used to refresh the states in the object (see Listing~\ref{lst:controller_update}). It simply sets the boolean values of the buttons according to the states given from the SPI driver.

\begin{lstlisting}[caption={Updating the state of the controller},label={lst:controller_update},frame=tlrb, language=C++]{Name}
void update() {
    uint8_t data = base::read();
		a      = get(data, ButtonOrder::A);
		b      = get(data, ButtonOrder::B);
		select = get(data, ButtonOrder::Select);
		start  = get(data, ButtonOrder::Start);
		up     = get(data, ButtonOrder::Up);
		down   = get(data, ButtonOrder::Down);
		left   = get(data, ButtonOrder::Left);
		right  = get(data, ButtonOrder::Right);
}

inline bool get(const uint8_t& data, ButtonOrder button) {
    return !!(data & (1 << button));
}
\end{lstlisting}

