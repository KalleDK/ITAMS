\subsection{Field}

When the Snake moves around it moves onto Fields. A Field can be either a Snake, None, Border or Fruit. To keep the Field's footprint small we use a uin8\_t to keep both the type and the values. So to correctly get the information from a Field, you first have to get the Type Id and then the appropriate function to get the values

\begin{lstlisting}[caption={Getting Correct Data From a Field},label={lst:get_snake},frame=tlrb, language=C++]{Name}
if (field->get_id() == type_id::Snake) {
  snake_value tmp = field->get_snake();
  // Do Stuff
}
\end{lstlisting}

This is implemented by simply letting the first 4 bits being the type id and the next 4 the values from this type.
