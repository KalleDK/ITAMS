\subsection{Field}

When the Snake moves around it moves onto Fields. A Field can be either a Snake, None, Border or Fruit. To keep the Field's footprint small we use a \texttt{uin8\_t} to keep both the type and the values. So to correctly get the information from a Field, you first have to get the type id and then call the appropriate function to get the values.

\begin{lstlisting}[caption={Getting correct data from a field},label={lst:get_snake},frame=tlrb, language=C++]{Name}
if (field->get_id() == type_id::Snake) {
    snake_value tmp = field->get_snake();
    // Do Stuff
}
\end{lstlisting}

This is implemented by simply letting the first four bits (LSB first) being the type id and the next four the values from this type.

So when you want to set a Field to a fruit you use the \texttt{set\_fruit} method. This calls the set method, that is private (see Listing~\ref{lst:set_fruit}). This is to ensure that you can't by accident create a Field with a Snake type id, but a Fruit's value.

\begin{lstlisting}[caption={Updating a Field},label={lst:set_fruit},frame=tlrb, language=C++]{Name}
public:
void set_fruit(fruit_value value) {
    set(type_id::Fruit, static_cast<uint8_t>(value));
}
private:
void set(type_id id, uint8_t value) {
		data = value << 4 | static_cast<uint8_t>(id);
}
\end{lstlisting}
