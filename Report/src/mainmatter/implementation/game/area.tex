\subsection{Area}

Now we have a Field that represents a part of the whole playing surface. To keep the Fields in order, we have the Area class. This class holds an array of fields. It has convenient methods for getting a specific field when playing. It could be that we need to check that the field the Snake moves into, is not a collision, thats why we have the method in Listing~\ref{lst:get_neighbour}.


\begin{lstlisting}[caption={Getting a neighbour field},label={lst:get_neighbour},frame=tlrb, language=C++]{Name}
value_type_ptr get_neighbour(value_type_ptr field, Direction direction) {
	switch(direction) {
		case(Direction::Up) : return field - width;
		case(Direction::Down) : return field + width;
		case(Direction::Left) : return field - 1;
		case(Direction::Right) : return field + 1;
		default: return field;
	}
}
\end{lstlisting}

The Area class is a nice encapsulation of the playing field, which contains a few functions to use such as finding out which field is a neighbor.