\subsection{Area}

Now we have a field that represents a part of the whole playing surface, to keep the fields in order we have the area class.
This class holds an array of fields, it also has convinience methods for getting a specific field for using when playing.
It could be that we need to check that the field the snake moves into is not a collision, thats why we have the method in figure \ref{lst:get_neighbour}


\begin{lstlisting}[caption={Getting a neighbour field},label={lst:get_neighbour},frame=tlrb, language=C++]{Name}
value_type_ptr get_neighbour(value_type_ptr field, Direction direction) {
	switch(direction) {
		case(Direction::Up) : return field - width;
		case(Direction::Down) : return field + width;
		case(Direction::Left) : return field - 1;
		case(Direction::Right) : return field + 1;
		default: return field;
	}
}
\end{lstlisting}

The area class is a nice encapsulation of the playing field which contains a few functions to calculate use full information such as which field is a neighbour.