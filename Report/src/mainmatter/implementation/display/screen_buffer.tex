\subsection{Screen buffer}

Instead of constantly calculating how the screen looks, we have a buffer that keeps a copy of the current state of the screen. This enables us to do calculations easier, and only update parts of the screen as needed. 
To save space, the buffer keeps a 8 pixel state per \texttt{uint8\_t}.

\subsubsection{Setup}

The setup of this buffer is performed automatically via templates on compile time. The only requirements there are to the Display driver is the following variables and methods. This enables the buffer to be used with other Displays.

\begin{itemize}
\item \texttt{SetAddress}
\item \texttt{Write}
\item \texttt{rows}
\item \texttt{columns}
\item \texttt{addressing\_static}
\item \texttt{addressing\_vertical}
\item \texttt{current\_addressing\_vertical} (only needed if \texttt{addressing\_static} is false)
\end{itemize}

The two first methods are used to set the row and column to where we want to write on the Display and what to actually write there.

The next two are used to create a correctly sized array on compile time thereby eliminating any dynamic allocation.

As many displays can either work in vertical mode or horizontal mode (see Figure~\ref{fig:array_horizontal} and Figure~\ref{fig:array_vertical}) or change dynamically between the two modes, the last two/three variables are used to optimize the communication to the Display.

\begin{figure}
\centering
\begin{minipage}{.5\textwidth}
  \centering
  \includegraphics[width=.4\textwidth]{implementation/array_horizontal}
  \caption{Horizontal Addressed Display}
  \label{fig:array_horizontal}
\end{minipage}%
\begin{minipage}{.5\textwidth}
  \centering
  \includegraphics[width=.4\textwidth]{implementation/array_vertical}
  \caption{Vertical Addressed Display}
  \label{fig:array_vertical}
\end{minipage}
\end{figure}

If the Display can shift between vertical and horizontal mode, then we need a runtime check on the \texttt{current\_addressing\_vertical} to be sure we are writing the correct way (see Listing~\ref{lst:dynamic_mode}). Now if the Display only works in one direction, then there's no need to have this runtime check. So if \texttt{addressing\_static} is true on compile time, only the correct direction is compiled into the binary and used directly (see Listing~\ref{lst:static_vertical}).

\noindent\begin{minipage}[t]{.45\textwidth}
\begin{lstlisting}[caption={Dynamic Mode},label={lst:dynamic_mode},frame=tlrb, language=C++]{Name}
template<typename Q = DISPLAY>
typename enable_if<!Q::addressing_static, void>::type
update(Point point, uint8_t width, uint8_t height) {
    if (display->current_addressing_vertical) {
  	    update_vertical(point, width, height);
  	} else {
  			update_horizontal(point, width, height);
  	}
}
\end{lstlisting}
\end{minipage}\hfill
\begin{minipage}[t]{.45\textwidth}
\begin{lstlisting}[caption={Static Vertical Mode},label={lst:static_vertical},frame=tlrb, language=C++]{Name}
template<typename Q = DISPLAY>
typename enable_if<Q::addressing_static && Q::addressing_vertical_mode, void>::type
update(Point point, uint8_t width, uint8_t height) {
    update_vertical(point, width, height);
}
\end{lstlisting}
\end{minipage}

\subsubsection{Usage}

The user of the buffer should not in general worry about how the data is stored in the 8 pixel's per \texttt{uint8\_t} array. The only function that allows the user to access the array directly is \texttt{set\_data}, which is used for text that matches the 8-bit height.

The rest of the function uses pixel x,y coordinates. The two main functions are \texttt{draw\_square} and \texttt{clear\_square} as most figures can be made with these. They calculate/recalculate the dimension given, to make sure only the part inside the screen is added. It also does the hard merging part of already existing data and the new data, when they share the same \texttt{uint8\_t}.

